\documentclass[]{article}
\usepackage{lmodern}
\usepackage{amssymb,amsmath}
\usepackage{ifxetex,ifluatex}
\usepackage{fixltx2e} % provides \textsubscript
\ifnum 0\ifxetex 1\fi\ifluatex 1\fi=0 % if pdftex
  \usepackage[T1]{fontenc}
  \usepackage[utf8]{inputenc}
\else % if luatex or xelatex
  \ifxetex
    \usepackage{mathspec}
    \usepackage{xltxtra,xunicode}
  \else
    \usepackage{fontspec}
  \fi
  \defaultfontfeatures{Mapping=tex-text,Scale=MatchLowercase}
  \newcommand{\euro}{€}
\fi
% use upquote if available, for straight quotes in verbatim environments
\IfFileExists{upquote.sty}{\usepackage{upquote}}{}
% use microtype if available
\IfFileExists{microtype.sty}{%
\usepackage{microtype}
\UseMicrotypeSet[protrusion]{basicmath} % disable protrusion for tt fonts
}{}
\usepackage[margin=1in]{geometry}
\ifxetex
  \usepackage[setpagesize=false, % page size defined by xetex
              unicode=false, % unicode breaks when used with xetex
              xetex]{hyperref}
\else
  \usepackage[unicode=true]{hyperref}
\fi
\hypersetup{breaklinks=true,
            bookmarks=true,
            pdfauthor={},
            pdftitle={},
            colorlinks=true,
            citecolor=blue,
            urlcolor=blue,
            linkcolor=magenta,
            pdfborder={0 0 0}}
\urlstyle{same}  % don't use monospace font for urls
\setlength{\parindent}{0pt}
\setlength{\parskip}{6pt plus 2pt minus 1pt}
\setlength{\emergencystretch}{3em}  % prevent overfull lines
\setcounter{secnumdepth}{0}

%%% Change title format to be more compact
\usepackage{titling}
\setlength{\droptitle}{-2em}
  \title{}
  \pretitle{\vspace{\droptitle}}
  \posttitle{}
  \author{}
  \preauthor{}\postauthor{}
  \date{}
  \predate{}\postdate{}




\begin{document}

\maketitle


\section{\texorpdfstring{\textbf{Paul D.
Blischak}}{Paul D. Blischak}}\label{paul-d.-blischak}

\emph{456 Aronoff Laboratory, 318 W. 12th Ave., Columbus, OH 43210}

Email: \href{mailto:blischak.4@osu.edu}{\nolinkurl{blischak.4@osu.edu}}

Website:
\href{http://pblischak.github.io}{\url{http://pblischak.github.io}}

GitHub:
\href{https://github.com/pblischak}{\url{https://github.com/pblischak}}

\subsection{\texorpdfstring{\textbf{Education}}{Education}}\label{education}

\textbf{In progress} -- Ph.D., Evol., Ecol., \& Org. Biology (EEOB), The
Ohio State University (OSU). Graduate minor in Statistics (completed).

\textbf{2012} -- B.Sc., Mathematics, The Ohio State University. Minors:
Statistics, Spanish.

\subsection{\texorpdfstring{\textbf{Professional
Experience}}{Professional Experience}}\label{professional-experience}

\textbf{2015--current} -- Graduate Research Associate, Dept. of EEOB,
OSU.

\textbf{2014} -- Teaching Associate, MBI Undergraduate Summer Program
(Phylogenetics Lab), OSU.

\textbf{2013--2015} -- Graduate Teaching Associate, Dept. of EEOB, OSU.

\textbf{2012--2013} -- Graduate Student Fellow, Dept. of EEOB, OSU.

\textbf{2011--2012} -- Undergraduate Research fellow, RUMBA Program,
OSU.

\subsection{\texorpdfstring{\textbf{Publications}}{Publications}}\label{publications}

\begin{enumerate}
\def\labelenumi{\arabic{enumi}.}
\setcounter{enumi}{1}
\item
  \textbf{Blischak, P. D.}, L. S. Kubatko and A. D. Wolfe. 2015.
  Accounting for genotype uncertainty in the estimation of allele
  frequencies in autopolyploids. \emph{In review}. bioRxiv, doi:
  \href{http://dx.doi.org/10.1101/021907}{\url{http://dx.doi.org/10.1101/021907}}.
\item
  \textbf{Blischak, P. D.}, A. J. Wenzel and A. D. Wolfe. 2014. Gene
  prediction and annotation in \textit{Penstemon} (Plantaginaceae): a
  workflow for marker development from extremely low-coverage genome
  sequencing. \emph{Applications in Plant Sciences} 2: 1400044. doi:
  \href{http://dx.doi.org/10.3732/apps.1400044}{\url{http://dx.doi.org/10.3732/apps.1400044}}.
\end{enumerate}

\subsection{\texorpdfstring{\textbf{Presentations}}{Presentations}}\label{presentations}

(*presenting author, \(^{\ddagger}\)undergraduate author)

\begin{enumerate}
\def\labelenumi{\arabic{enumi}.}
\setcounter{enumi}{8}
\item
  \textbf{Blischak, P. D.}*, L. S. Kubatko and A. D. Wolfe. July 2015.
  Estimating allele frequencies in non-model autopolyploids using high
  throughput sequencing data. Botany 2015. Edmonton, Alberta.
  fig\textbf{share}, doi:
  \href{http://dx.doi.org/10.6084/m9.figshare.1495514}{\url{http://dx.doi.org/10.6084/m9.figshare.1495514}}.
\item
  Wolfe, A. D.*, B. Stone\(^{\ddagger}\), N. Padmalwar\(^{\ddagger}\),
  \textbf{P. D. Blischak} and L. S. Kubatko. July 2015. \emph{Hyobanche
  sanguinea} (Orobanchaceae): there's more than meets the eye. Botany
  2015. Edmonton, Alberta.
\item
  Zacarias-Correa, A. G., A. J. Wenzel, \textbf{P. D. Blischak}, A. D.
  Wolfe*, E. P. Calix and M.-S. Perez. July 2015. Molecular phylogeny of
  \emph{Penstemon} section \emph{Fasciculus} (Plantaginaceae) based on
  single copy orthologous genes (COSII). Botany 2015. Edmonton, Alberta.
\item
  \textbf{Blischak, P. D.}* More genomes, more problems. February 2015.
  Graduate Student Forum Speed-Talks, OSU.
\item
  \textbf{Blischak, P. D.}*, A. D. Wolfe and L. S. Kubatko. October
  2014. Phylogenomics and the coalescent model: new tools for new data.
  EEOB Graduate Student Talks, OSU.
\item
  \textbf{Blischak, P. D.}*, A. D. Wolfe and L. S. Kubatko. July 2014.
  Inferring large phylogenies under the coalescent model using SNPs from
  next-generation sequence data. Botany 2014: Boise, ID.
  fig\textbf{share}, doi:
  \href{http://dx.doi.org/10.6084/m9.figshare.1436072}{\url{http://dx.doi.org/10.6084/m9.figshare.1436072}}.
\item
  Wolfe, A. D.* and \textbf{P. D. Blischak}. July 2014. Patterns of
  diversity in \emph{Penstemon} (Plantaginaceae). Botany 2014: Boise,
  ID.
\item
  \textbf{Blischak, P. D.}*, A. J. Wenzel, M. R. Stevens and A. D.
  Wolfe. August 2013. How low can you go? Gene predictions and putative
  annotations in four species of \emph{Penstemon} (Plantaginaceae) from
  ultra low-coverage 454 sequencing. Botany 2013: New Orleans, LA.
\item
  Wenzel, A. J.*, \textbf{P. D. Blischak} and A. D. Wolfe. August 2013.
  Molecular phylogenetic analysis of \emph{Penstemon} (Plantaginaceae)
  section \emph{Ericopsis}: evaluating relationships and taxonomic
  affinities. Botany 2013: New Orleans, LA.
\end{enumerate}

\subsection{\texorpdfstring{\textbf{Software}}{Software}}\label{software}

\textbf{polyfreqs}: an R package for the estimation of allele
frequencies in autopolyploids. Available on GitHub:
\href{https://github.com/pblischak/polyfreqs}{\url{https://github.com/pblischak/polyfreqs}}.

\subsection{\texorpdfstring{\textbf{Honors and
Awards}}{Honors and Awards}}\label{honors-and-awards}

\textbf{2015} -- ASPT Graduate Student Research Grant, \$800.00.

\textbf{2014} -- SSB Graduate Student Research Award, \$1915.00.

\textbf{2014} -- NSF Graduate Research Fellowship, Honorable Mention.

\textbf{2014} -- NIMBioS Visiting Graduate Student Fellowship (advised
by Dr.~Brian O'Meara).

\textbf{2013} -- Beatley Award for Field Work in Plant Systematics, OSU
Herbarium, \$1050.00.

\textbf{2012} -- Distinguished University Fellowship, OSU Graduate
School.

\textbf{2011} -- Undergraduate Research Fellowship, RUMBA Program, OSU.

\subsection{\texorpdfstring{\textbf{Service}}{Service}}\label{service}

\textbf{2014} -- Graduate Student Representative: Diversity Committee.

\textbf{2013} -- Graduate Student Representative: Seminar Committee.

\textbf{Journal reviews} -- \emph{Molecular Phylogenetics and
Evolution}.

\subsection{\texorpdfstring{\textbf{Membership}}{Membership}}\label{membership}

\textbf{2013-current} -- American Penstemon Society (APS), American
Society of Plant Taxonomists (ASPT), Botanical Society of America (BSA),
Society of Systematic Biologists (SSB), Society for the Study of
Evolution (SSE).

\begin{center}\rule{0.5\linewidth}{\linethickness}\end{center}

Last updated: 28 July, 2015

\end{document}
